\chapter{Conclusions and Discussion} % discussion and future work

Provenance is a type of metadata representing the lineage of a digital object. In the field of personal data management this could be used to help users understand how their personal data is used, an issue that is particulary relevant as we store more and more information about ourselves. The problem is that raw provenance is too difficult to understand and the current industry standard for visualising it (directed acyclic graphs) is still too complex for non technical users (and in a lot of cases technical users). 

To solve this problem I designed two methods for manually clustering nodes, \textit{ctrl+click} and search clustering. Clustering could then be used to simplify provenance graphs in order to hide irrelevant information or focus attention on certain areas of a provenance graph, for example how a single users information was used in a report. I then implemented a working prototype so that a usability study could be conducted on the usability and effectiveness of each clustering technique. Using both a thing aloud study (the gold standard of usability feedback) and a SUS questionnaire (one of the most widely used questionnaires) I conducted a comprehensive usability study on the ProvOwl application. 

The results from the usability study showed that after been prompted to the existence of the clustering function users would use both the \textit{ctrl+click} and search clustering methods in order to simplify graphs. After a small amount of time users were comfortable and confident using the interface to modify graphs to convey meaning, indicating that ProvOwl is a usable interface for clustering graphs.



\section{Future Work}

In the future it would be useful to expand on some of ProvOwl's features as well as implementing some of the other clustering techniques recommended by users in the think aloud study. I would then propose to re-conduct the usability studies and in particular the SUS questionnaire to see if the same divide in results appears.

Future challenges involve allowing users to search for all nodes that match one regex but not another. It would also be useful to only match searches on particular properties of a node, for example a user may only want to match their search on \textit{author} of a node. Many users during usability studying requested the ability to cluster all the children of a node, extending from this it would also be useful to allows user to cluster based on a nodes relationships with other nodes, for example ``Cluster all nodes that match the following regex \texttt{X-tweets|query-X-Time} and are not derived from \texttt{TwitterFeed-time-3}. As mentioned in the think aloud results (\ref{sub:think_alouds}) it would also be useful to have a feature that lets users draw a box over the nodes they want to cluster.

In the future I would also like to expand on the export abilities of the prov owl application to allow exporting in the PROV-N standard. This would allow sharing of provenance graphs with manually created clusters. This would also xxxx expanding the current provenance standard to include syntax for representing nodes that are clustered together.

Lastly, server side computation may be a useful feature for the future, a server with higher computational power and resources may be able to find and analyse interesting details about a file. This could also be used to explore incredibly large provenance files. By having the graph loaded server side and only sections required sent to the client, the computational load on the client could be greatly reduced.
