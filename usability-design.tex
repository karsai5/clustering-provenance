\chapter{Usability study design}

Provenance graphs can be quite large. They also hold and present information in a variety of aspects: \textit{nodes}, \textit{relationships} and \textit{attributes}. Node data is the graph information related to a nodes: the name of a node as well as its type (eg. entity, agent). Relationship data is information implied through connections; an edge between two nodes represents some sort of relationship between them, whether it be that one \textit{used} the other or that an entity was generated by another node. Attribute data is the properties given to a node, usually generated during collection. For example, a node representing a stream of twitter data may have a start and end property specify the times that the feed covers\footnote{Because provenance presents data in three different aspects (nodes, relationships and attributes), It becomes difficult to secure because a security model would have to take into account the three different types of data\cite{Braun2008}}.

Because of this large and heterogeneous amount of information stored in provenance graphs I found it important that an interface allowed users to interactively explore the graph rather than just using static images of provenance graphs. It is because of this that I decided to undertake my usability study using a prototype interface that allows clustering of nodes. 

\section{Interview format}
\label{sec:interview_format}

Interviews were undertaked with participants in order to guage how effective clustering is as a form of graph simplification and a tool for representing information. The participants were asked to complete a series of expercises using the ProvOwl interface whilst I took notes. They undertook these exercises on my laptop with the browser in full-screen mode in order to limit distractions. A mouse was provided in case users hadn't used a MacBook trackpad before. The interviews were undertaken wherever was most convenient to the participant, this included at my desk, participant's houses or coffee shops.

The study was split into four sections: background information, provenance primer, three exercises and a SUS questionnaire. 

\subsection{Background Information}
\label{sec:background_information}

General background information was collected about each participant. Gender, age groups as well as highest level of education. These were recorded in order to identify any demographic skew that I may have in my cohort as well as in case any unexpected correlations occurred. As a last general question I also asked if they had heard of the term \textit{provenance} before, and if so to explain what it meant. Because provenance is used in other fields (such as accounting and art dealership) I wanted to identify anyone who had previous concepts of provenance and lineage to identify if this effected their ability to interpret provenance graphs.

\subsection{Provenance Primer}
\label{sec:provenance_primer}

After the participant had answered the general background questions they were given a information sheet explaining the concept of provenance in breif and showing an example of a provenance graph as well as explanations of what the different coloured and shaped nodes meant (\ref{prov-primer}). This allowed participants to have a uniform understanding of the concept of provenance reguardless of whether they had heard of it before or not. Notes were taken of any questions the participant had whilst reading the information sheet.

\subsection{Exercises}
\label{sec:exercises}

\begin{figure}[b]
\caption{Scenario for exercise one. View text in context in Appendix TODO}
\begin{framed}
\begin{flushleft}
Suppose you are Alice. You have been using your FitBit and Withings scales for 3 years to reduce your weight to 61kg and to increase your physical activity to 8000 steps a day. You have just linked your FitBit account to two friends, Bob and Carol. Recently FitBit introduced a fitness score that is shared and ranked with your friends\textellipsis This makes you wonder just how your score has been calculated.
\end{flushleft}
\end{framed}
\end{figure}
\begin{figure}[b]
\caption{Scenario for exercise two. View text in context in Appendix TODO}
\begin{framed}
\begin{flushleft}
You are a researcher creating a citizen-data-report that recommends strategies for improving the fitness of your local community. You have step, location, weight and calorie information about members of the community. You are using this information to create multiple reports regarding fitness and weight information which will be used to support a final report addressed to the community on strategies that can be used to improve overall fitness. 
\end{flushleft}
\end{framed}
\end{figure}
\begin{figure}[b]
\caption{Scenario for exercise three. View text in context in Appendix TODO}
\begin{framed}
\begin{flushleft}
This provenance graph is an abstract representation of a report based on twitter data. 
\end{flushleft}
\end{framed}
\end{figure}

Overall there was three exercises. Each exercise described a particular scenario assosiated with a provenance file and required users to answer questions and complete tasks related to that provenance file. 

For example, in the first exercise the user is asked to imagine that they're Alice, they track their fitness information with a Fitbit for steps and a Withings scale for weight (you can find the exact example in \ref{interview-questionnaire}). The first question of each exercise was a simple identification query. The participant would be asked to identify different aspects of the graph or to trace the lineage from a certain element. This allowed the participants to take their time when first analysing the graph as well as giving me an understanding of their mental model.

Each exercise was progressively more conceptually difficult than the previous. The first, as mentioned before, invovled putting yourself in the shoes of Alice and answering a series of questions related to your data. The provenance graph they are given to explore starts with nodes detailing their weight and steps clustered respectively. This meant that users were required to ungroup the nodes either by clicking the ungroup button from a menu or double clicking on nodes (most users identified that double clicking would un-cluster nodes without prompting) in order to correctly answer the questions. This first exercise was completely exploration based and no modifications of the graph where required.

The second exercise invovled a larger amount of data and asked participants to imagine themselves as the curator of a citizen science project. Data was been tracked about multiple users and then used in a report that gave feedback on how to improve the cohorts overall fitness. Similar to exercise one there was first a task asking the user to describe the contents of the graph to me. Following from there was two tasks that required the user cluster nodes in order to hide information. Because un-clustering was required to complete the first task users where accustomed to the concept of clustering nodes by the time they got to this task.


The third and final exercise expanded on the previous by requiring users to convey information by modifying the graph. This is silghtly different to exercise two which required the participant to \textit{hide} information, instead users had to convey a certain point through a graph. Same as above the first task asked the user to describe the graph, a bit more difficult as the graph was abstract compared to the previous two. The following two tasks where of the following format: \textit{You want to modify the graph in order to convey the following information: <information to convey>}. 

\subsection{SUS Questionnaire}
\label{sec:sus_questionnaire}

% TODO: Insert references

In order to test the effectiveness of a feature of an interface I thought it was important also to gauge how effective the interface was on a whole. The System Usability Scale has been around since 1986. It provides a quick easy way of identifying if an interface is easy to use or not. 

The SUS questionnaire consists of 10 questions each requiring a checkbox response on a scale of 1-5 stating if they strongly agree of strongly disagree. Users are asked to reply with what they intuitively feel rather than thinking about the questions for an extended period. 

\section{Coded questions}
\label{sec:coded_questions}

Each question was given a set of points, user actions that could be checked off if a user did them. For example, in question one of exercise one, the participant was asked \textit{What infroation is used to calculate your [fitness] score?}. One of the points for this question was \textit{ungrouped nodes}, so if the user ungrouped nodes while answering the question this point was ticked off. Whilst hand written notes were also taken during interviews, these checkboxes made for an easy way of identitying what common actions users undertook to complete exercises. 

Each point was assigned a code identifying what class of activity it belonged to. In the above example the ungrouping point had the code \textit{grouping-ungroup} because it's completing a grouping function and in particular an ungrouping function. 

Each of the codes are listed in Table~\ref{tab:question-codes} with an explanation of the user behaviour they map to.

\begin{table}[h]
	\centering
	\caption{List of question codes and descriptions of each.}
		\def\arraystretch{1.5}
	\label{tab:question-codes}
	\begin{tabu} to \textwidth { | l | X[l] | }
		\hline
		\textbf{Code}& \textbf{Description}  \\	
		\hline
		\hline
		grouping & Activities that involved users grouping or ungrouping nodes (by any means)\\
		\hline
		grouping-group &  When a user grouped a node \\
		grouping-ungroup & When a user ungrouped a node \\
		grouping-complex & When a user grouped a complex set of noes \\
		\hline
		\hline
		graph & Activities related to understanding a graph conceptually \\
		\hline
		graph-explain & Notice and comment on information in the graph. An example is in exercise one where some users noticed that weight data is both manually and automatically logged. This information is not usually relevant to the question. \\
		graph-understanding & Correctly understand a concept from the graph, usually directly in response to a question. \\
		\hline
		\hline
		lineage & Correctly identifying information related to the lienage of a node. \\
		\hline
		lineage-identify & Correctly identified simple lineage. \\
		lineage-implied & Correctly identified lineage passing through multiple entities. \\
		lineage-complex & Correctly identified lineage that was complex, usually identifying multiple nodes. \\
		\hline
		\hline
		interface & When a user uses a specific bit of the ProvOwl interface. \\
		\hline
		interface-export-image & Using the export -> image menu item to save an image of a graph.\\
		interface-os-screenshot & Using the operating system screenshot functionality to save an image of a graph. \\
		interface-rename & Using the rename function (found in the contexual links in a nodes details) to rename a node. \\
		interface-manual-group & Manually group nodes by ctrl+clicking multiple nodes and selecting \textit{group nodes} from the contexual menu. \\
		interface-search-group & Grouping nodes by searching for the nodes they want to use using the search function and then grouping them. \\
		\hline
		\hline
		privacy & Whether a user was privacy concious at all in relation to personal data. \\
		\hline
 
	\end{tabu}
\end{table}


