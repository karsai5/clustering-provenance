\chapter{Future Work}
The prototype source code is available on GitHub\footnote{\url{https://github.com/karsai5/ProvOwl}}. The interface is also available online where you can test it with a sample graph\footnote{\url{http://provowl.com/?file=/prov-examples/provn/threenode.provn}}.
We describe features under development to improve usability.

We propose to create a regular expression language to select nodes, as mentioned in the section on challenges. 
This will enable users to select nodes in cases such as:
(i) Select all nodes with the name ``*Feed''
(ii) Select all nodes \textit{not} influencing the ``Summarize'' node
(iii) Select all children of ``Fitness-Summary''.
For large graphs, this could allow for faster user-directed simplification of a graph. 

As an extension of this, the language should describe parametered clustering, so users ask for multiple similar clusters to be formed. For example ``Create clusters from nodes the same depth from the root'' or even using data inside the nodes ``Create clusters from nodes that have the same creation date''.  

This could also impact the way nodes are automatically named. If the user created clusters from nodes that all have the same creation date, the system may infer that the name for the new node should include the creation date.

We also wish to extend the PROV standard to include descriptions of cluster nodes. This would allow a user to cluster nodes manually, export the PROV file with cluster descriptions and then share with someone else. This \textit{extraction}, being able to export your current state of exploration, would allow further exploration from the current state later on or even sharing with other users for further analysis.
